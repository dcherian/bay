
% Here are a few math abbreviations that may be useful. Note that math
% variables should generally be set in italic, vectors in bold italic, and
% constants, such as e and i should be upright. Operators like the 'd' in
% dx  and 'D' in D/Dt should also be upright.
\newcommand{\ir}{\mathrm{i}}
\newcommand{\e}{\, \mathrm{e}} \newcommand{\er}{\mathrm{e}}

% The following makes ODEs and PDEs easier to write.
% For an example, see the second problem below. - Vallis Solution template
\newcommand{\dr}{\mathop{}\!\mathrm{d}}
\newcommand{\D}{\mathop{}\!\mathrm{D}}
\newcommand{\dd}[3][]{{\frac{\dr^{#1} #2}{\dr #3^{#1}}}}
\newcommand{\pp}[3][]{{\frac{\partial^{#1} #2}{\partial #3^{#1}}}}
\newcommand{\DD}[1]{{\frac{\D#1}{\D t}}}
% example: \pp[2]y x

\newcommand\ppx{\pp{}{x}}
\newcommand\ppy{\pp{}{y}}
\newcommand\ppt{\pp{}{t}}

% absolute value signs
% https://tex.stackexchange.com/questions/43008/absolute-value-symbols#
\DeclarePairedDelimiter\abs{\lvert}{\rvert}%
\DeclarePairedDelimiter\norm{\lVert}{\rVert}%

% Swap the definition of \abs* and \norm*, so that \abs
% and \norm resizes the size of the brackets, and the
% starred version does not.
\makeatletter
\let\oldabs\abs
\def\abs{\@ifstar{\oldabs}{\oldabs*}}
%
\let\oldnorm\norm
\def\norm{\@ifstar{\oldnorm}{\oldnorm*}}
\makeatother

% These are some of my math shortcuts
\newcommand{\mb}[1]{\boldsymbol{#1}} % math bold font
\newcommand{\ol}[1]{\overline{#1}}
\newcommand{\x}{\times}
\newcommand{\mO}{\textnormal{O}}
\newcommand{\ubr}{\underbracket}
\newcommand{\disp}{\displaystyle}
\newcommand{\vb}{\mb {v}\xspace}
\newcommand{\ub}{\mb {u}\xspace}
\mathchardef\mhyph="2D

\DeclareMathOperator{\erf}{erf}

% non-dimensional numbers
\newcommand{\Ro}{\textnormal{Ro}}
\newcommand{\Rot}{\textnormal{Ro}_\textnormal{T}}
\newcommand{\Rh}{\textnormal{Rh}}
\newcommand{\Ri}{\textnormal{Ri}}
\newcommand{\Rif}{\Ri_f}
\newcommand{\Pra}{\textnormal{Pr}}
\newcommand{\Pe}{\textnormal{Pe}}
\newcommand{\Ra}{\textnormal{Ra}}
\newcommand{\Rey}{\textnormal{Re}}
\newcommand{\Fr}{\textnormal{Fr}}
\newcommand{\Bu}{\textnormal{Bu}}

\newcommand\Hsb{H_\text{sb}}
\newcommand\tnd{t_\text{nd}}
\newcommand\bt{_\text{bot}}

\newcommand{\boxedeq}[2]{\begin{empheq}[box={\fboxsep=6pt\fbox}]{equation}{\label{#1}#2}\end{empheq}}

\newcommand{\boxedalign}[2]{\begin{empheq}[box={\fboxsep=6pt\fbox}]{align}\label{#1}#2\end{empheq}}
